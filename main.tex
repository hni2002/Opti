\documentclass{article}
\usepackage{graphicx} % Required for inserting images
\usepackage{matlab-prettifier}
\usepackage{amsmath}
\usepackage{amssymb}
\usepackage{textcomp,xcolor}
\definecolor{MatlabCellColour}{RGB}{252,251,220}
\usepackage{listings}
\definecolor{mygreen}{rgb}{0,0.6,0}
\definecolor{mygray}{rgb}{0.5,0.5,0.5}
\definecolor{mymauve}{rgb}{0.58,0,0.82}

\lstset{ 
	backgroundcolor=\color{white},   % choose the background color; you must add \usepackage{color} or \usepackage{xcolor}; should come as last argument
	basicstyle=\footnotesize,        % the size of the fonts that are used for the code
	breakatwhitespace=false,         % sets if automatic breaks should only happen at whitespace
	breaklines=true,                 % sets automatic line breaking
	captionpos=b,                    % sets the caption-position to bottom
	commentstyle=\color{mygreen},    % comment style
	deletekeywords={...},            % if you want to delete keywords from the given language
	escapeinside={\%*}{*)},          % if you want to add LaTeX within your code
	extendedchars=true,              % lets you use non-ASCII characters; for 8-bits encodings only, does not work with UTF-8
	frame=single,	                   % adds a frame around the code
	keepspaces=true,                 % keeps spaces in text, useful for keeping indentation of code (possibly needs columns=flexible)
	keywordstyle=\color{blue},       % keyword style
	language=Matlab,                 % the language of the code
	morekeywords={*,...},            % if you want to add more keywords to the set
	numbers=left,                    % where to put the line-numbers; possible values are (none, left, right)
	numbersep=5pt,                   % how far the line-numbers are from the code
	numberstyle=\tiny\color{mygray}, % the style that is used for the line-numbers
	rulecolor=\color{black},         % if not set, the frame-color may be changed on line-breaks within not-black text (e.g. comments (green here))
	showspaces=false,                % show spaces everywhere adding particular underscores; it overrides 'showstringspaces'
	showstringspaces=false,          % underline spaces within strings only
	showtabs=false,                  % show tabs within strings adding particular underscores
	stepnumber=2,                    % the step between two line-numbers. If it's 1, each line will be numbered
	stringstyle=\color{mymauve},     % string literal style
	tabsize=4,	                   % sets default tabsize to 2 spaces
	title=\lstname                   % show the filename of files included with \lstinputlisting; also try caption instead of title
}

\title{Optimization Practical Exercise \\ Sommersemester 2023}
\author{Donnermair Maximilian @students.jku.at\\ Fromherz Jakob @students.jku.at\\Haslhofer Eva-Maria  k12007773@students.jku.at \\ Scharnreitner Franz @students.jku.at\\ Weiss Hannah k12021111@students.jku.at } %Bitte Matrikelnummer in Email einfügen
\date{12 May 2023}

%Bitte im Folder Code die Programmfiles am Schluss aktualisieren
\begin{document}
	
	\maketitle
	
	\newpage
	
	\section{Exercise 1}
	
	\subsection{Linesearch Algorithm with Wolfe-Powell Condition}
	\lstinputlisting{src/LineSearch.m}
	
	\subsection{Method of steepest descent}
	\lstinputlisting{src/SteepestDescent.m}
	
	
	\subsection{Testfunctions a) to d)}
	\lstinputlisting{src/f_a.m}
	
	\lstinputlisting{src/f_b.m}
	
	\lstinputlisting{src/f_c.m}
	
	\lstinputlisting{src/f_d.m}
	
	\subsection{Testscript}
	
	\lstinputlisting{src/testex1.m}
	With a quick printing function.
	\lstinputlisting{src/displayVals.m}
	
	We get the output
	\lstinputlisting{src/test1.txt}
	\subsection{Interpretation}
	We can see, that the SteepestDescent works for f\_a relatively well, while to get a relatively good result for f\_b we already need $>4000$ iterations.
	f\_c works quite well, altough the number of iterations needed grows fast with increase in dimension size. The algorithm fails to work for f\_d however for big dimensions.
	
	\section{Exercise 2}
	\subsection{Testfunctions a) to d) with Hessian}
	\lstinputlisting{src/f_aH.m}
	\lstinputlisting{src/f_bH.m}
	\lstinputlisting{src/f_cH.m}
	\lstinputlisting{src/f_dH.m}
	
	\subsection{Tests in Exercise 2}
	\lstinputlisting{src/Prog1Ex2.m}
    \subsection{Test output}
	
	\subsection{Interpretation}
	
	\section{Exercise 3}
	\subsection{Interpretation}
	
	
	\section{Exercise 4}
	\subsection{Source Code}
	\lstinputlisting{src/Prog1Ex4.m}
	\lstinputlisting{src/test4_1.txt}
	\subsection{Solution of \texttt{fminsearch}}
	When applying \texttt{fminsearch} to the problem considering the given non-continuously differentiable function 
	\begin{equation*}
		\min_{x\in\mathbb{R}^2}x_1+10\max\{x_1^2+2x_2^2-1,0\}
	\end{equation*}
	we get that $x=(x_1,x_2)\approx(-0.9997,-0.0168)$ solves the above equation where the exact solution should correspond to the vector $\overline{x}=(x_1,x_2)=(-1,0)$. This implies a numerical error of $\Vert x - \overline{x}\Vert \approx 0.0168$ as can be obtained by the \texttt{MATLAB} source code in section 4.1.
	\subsection{Solution of \texttt{fminunc}}
	If instead of \texttt{fminsearch} the \texttt{MATLAB} command \texttt{fminunc} is applied to the same problem as in section 4.2 we get the result of $x=(x_1,x_2)\approx(-1.0000,0.0000)$. Consequently, we also get a numerical error much smaller in size and given by $\Vert x - \overline{x}\Vert \approx 4.3007\cdot 10^{-5}$.
	\subsection{Interpretation}
	In the \texttt{MATLAB} output it states that \texttt{fminsearch} stopped because it cannot decrease the objective function along the current search direction any further. Though, if stopping criteria details are displayed we get the following additional information.
	\lstinputlisting{src/test4_2.txt}
	Consequently, it must be the case that the simplex search method of \texttt{fminsearch} which does not make use of numerical or analytic gradients as the line search algorithm in \texttt{fminunc} is not appropriate for the considered problem. On the opposite, \texttt{fminunc} estimates according gradients using finite differences and therefore provides an adequate line search interval and ultimately a more reliable result.
\end{document}
