\documentclass{article}
\usepackage{graphicx} % Required for inserting images
\usepackage{matlab-prettifier}

\usepackage{textcomp,xcolor}
\definecolor{MatlabCellColour}{RGB}{252,251,220}
\usepackage{listings}
\definecolor{mygreen}{rgb}{0,0.6,0}
\definecolor{mygray}{rgb}{0.5,0.5,0.5}
\definecolor{mymauve}{rgb}{0.58,0,0.82}

\lstset{ 
  backgroundcolor=\color{white},   % choose the background color; you must add \usepackage{color} or \usepackage{xcolor}; should come as last argument
  basicstyle=\footnotesize,        % the size of the fonts that are used for the code
  breakatwhitespace=false,         % sets if automatic breaks should only happen at whitespace
  breaklines=true,                 % sets automatic line breaking
  captionpos=b,                    % sets the caption-position to bottom
  commentstyle=\color{mygreen},    % comment style
  deletekeywords={...},            % if you want to delete keywords from the given language
  escapeinside={\%*}{*)},          % if you want to add LaTeX within your code
  extendedchars=true,              % lets you use non-ASCII characters; for 8-bits encodings only, does not work with UTF-8
  frame=single,	                   % adds a frame around the code
  keepspaces=true,                 % keeps spaces in text, useful for keeping indentation of code (possibly needs columns=flexible)
  keywordstyle=\color{blue},       % keyword style
  language=Matlab,                 % the language of the code
  morekeywords={*,...},            % if you want to add more keywords to the set
  numbers=left,                    % where to put the line-numbers; possible values are (none, left, right)
  numbersep=5pt,                   % how far the line-numbers are from the code
  numberstyle=\tiny\color{mygray}, % the style that is used for the line-numbers
  rulecolor=\color{black},         % if not set, the frame-color may be changed on line-breaks within not-black text (e.g. comments (green here))
  showspaces=false,                % show spaces everywhere adding particular underscores; it overrides 'showstringspaces'
  showstringspaces=false,          % underline spaces within strings only
  showtabs=false,                  % show tabs within strings adding particular underscores
  stepnumber=2,                    % the step between two line-numbers. If it's 1, each line will be numbered
  stringstyle=\color{mymauve},     % string literal style
  tabsize=4,	                   % sets default tabsize to 2 spaces
  title=\lstname                   % show the filename of files included with \lstinputlisting; also try caption instead of title
}

\title{Optimization Practical Exercise \\ Sommersemester 2023}
\author{Donnermair Maximilian @students.jku.at\\ Fromherz Jakob @students.jku.at\\Haslhofer Eva-Maria  k12007773@students.jku.at \\ Scharnreitner Franz @students.jku.at\\ Weiß Hannah @students.jku.at } %Bitte Matrikelnummer in Email einfügen
\date{May 2023}

%Bitte im Folder Code die Programmfiles am Schluss aktualisieren
\begin{document}

\maketitle

\newpage

\section{Exercise 1}

\subsection{Linesearch Algorithm with Wolfe-Powell Condition}
\lstinputlisting[frame=single, numbers=left, style=customc]{src/Linesearch.m}

\subsection{Method of steepest descent}
\lstinputlisting[frame=single, numbers=left, style=customc]{src/SteepestDescent.m}


\subsection{Testfunctions a) to d)}
\lstinputlisting[frame=single, numbers=left, style=customc]{src/f_a.m}

\lstinputlisting[frame=single, numbers=left, style=customc]{src/f_b.m}

\lstinputlisting[frame=single, numbers=left, style=customc]{src/f_c.m}

\lstinputlisting[frame=single, numbers=left, style=customc]{src/f_d.m}

\subsection{Testscript}

\lstinputlisting[frame=single, numbers=left, style=customc]{src/testex1.m}
With a quick printing function.
\lstinputlisting[frame=single, numbers=left, style=customc]{src/displayVals.m}

We get the output
\lstinputlisting[style=customc]{src/test1.txt}
\subsection{Interpretation}
We can see, that the SteepestDescent works for f\_a relatively well, while to get a relatively good result for f\_b we already need $>4000$ iterations.
f\_c works quite well, altough the number of iterations needed grows fast with increase in dimension size. The algorithm fails to work for f\_d however for big dimensions.

\section{Exercise 2}
\subsection{Testfunctions a) to d) with Hessian}
\lstinputlisting[frame=single, numbers=left, style=customc]{src/f_aH.m}
\lstinputlisting[frame=single, numbers=left, style=customc]{src/f_bH.m}
\lstinputlisting[frame=single, numbers=left, style=customc]{src/f_cH.m}
\lstinputlisting[frame=single, numbers=left, style=customc]{src/f_dH.m}

\subsection{Tests in Exercise 2}
\lstinputlisting[frame=single, numbers=left, style=customc]{src/Prog1Ex2.m}

\subsection{Interpretation}

\section{Exercise 3}
\subsection{Interpretation}


\section{Exercise 4}
\subsection{Interpretation}


\end{document}
